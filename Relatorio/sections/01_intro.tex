\section{Introdução}

\subsection{Contextualização}
%Contextualização do problema (sistema de automação com N células em cascata)

Este trabalho insere-se no âmbito da unidade curricular de Co-design e Sistemas Reconfiguráveis, focando-se no desenvolvimento e implementação de controladores para sistemas de automação industrial. O caso de estudo considerado é um sistema de produção composto por 3 células em cascata, onde cada célula integra um braço de robô e um tapete rolante.

A complexidade destes sistemas justifica a utilização de metodologias formais de modelação e verificação, permitindo validar o comportamento do controlador antes da sua implementação física. Adicionalmente, a possibilidade de implementar o controlador de forma centralizada ou distribuída oferece diferentes trade-offs entre complexidade de implementação, desempenho e escalabilidade do sistema.

\subsection{Objetivos}
%Objetivos do trabalho
O presente trabalho tem como objetivo principal o desenvolvimento completo de um controlador para um sistema de automação com três células em cascata (N=3), desde a fase de modelação até à implementação física em hardware reconfigurável.

Os objetivos específicos incluem:

\begin{itemize}
    \item Modelar o comportamento do sistema utilizando redes de Petri Input-Output Place-Transition (IOPT), explorando as capacidades de composição modular através de operações de adição e fusão de redes;
    
    \item Validar e analisar o modelo através de simulação e análise do espaço de estados, verificando propriedades comportamentais críticas do sistema;
    
    \item Implementar o controlador em plataformas de hardware distintas (FPGA e Arduino), avaliando a abordagem centralizada de controlo;
    
    \item Desenvolver uma arquitetura de controlo distribuído através da decomposição do modelo global, explorando paradigmas de execução síncrona e GALS (Globally Asynchronous Locally Synchronous);
    
    \item Analisar o impacto de comunicações não-instantâneas entre controladores distribuídos, introduzindo atrasos pseudo-aleatórios;
    
    \item Estender o modelo para suportar buffers de capacidade finita superior a uma unidade, aumentando a flexibilidade do sistema;
    
    \item Comparar as diferentes abordagens de implementação, avaliando vantagens, desvantagens e adequação a diferentes contextos aplicacionais.
\end{itemize}