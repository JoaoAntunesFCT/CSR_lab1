\section{Implementação em FPGA - Distribuição Síncrona}

\subsection{Geração do código VHDL}
%print código VHDL

\subsection{Implementação do LFSR}
O LFSR (Linear Feedback Shift Register) é um componente essencial para a geração de números pseudoaleatórios no sistema. A sua implementação em VHDL envolve a definição dos registos de shift (deslocamento) e das funções de feedback necessárias para garantir a pseudo-aleatoriedade dos valores gerados.
Usando como referência o documento fornecido pelos docentes, procedeu-se à implementação do LFSR com as características desejadas.

Esta secção detalha o processo de implementação do LFSR, incluindo a configuração dos seus parâmetros e a integração com os restantes componentes do sistema.
\subsubsection{Gestão dos atrasos}
%Integração dos delays

\subsection{Deployment na FPGA}
%baby steps do deployment na fpga

\subsection{Testes e análise de resultados}
%print testes
%comparar resultados