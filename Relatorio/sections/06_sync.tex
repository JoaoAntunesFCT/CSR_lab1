\section{Controlador Distribuído em regime Síncrono}

\subsection{Cutting Set}
A identificação do conjunto de corte (cutting set) é um passo crucial na decomposição de sistemas distribuídos. Este conjunto consiste em um grupo de canais ou conexões que, quando removidos, dividem o sistema em subsistemas independentes. A escolha adequada do cutting set permite a implementação eficiente de controladores distribuídos, minimizando a complexidade da comunicação entre os módulos.
%print da abordagem para cutting set

\subsection{Decomposição SPLIT}
A decomposição usando SPLIT (Synchronous Parallel Logic Interconnect Technology) permite a criação de uma arquitetura mais flexível e escalável, dividindo o sistema em módulos independentes que se comunicam através de canais síncronos. Essa abordagem facilita a implementação de sistemas complexos, permitindo a reutilização de componentes e a redução do tempo de desenvolvimento.
%print da decomposição SPLIT

\subsection{Modelo com canais síncronos}
A modelação com canais síncronos envolve a definição de canais de comunicação que operam em sincronia, permitindo a troca de informações entre os módulos de forma coordenada. Essa abordagem é fundamental para garantir a consistência e a previsibilidade no comportamento do sistema.
%print of that shit mm

\subsection{Modelo com canais assíncronos}
A modelação com canais assíncronos envolve a definição de canais de comunicação que operam de forma independente, permitindo a troca de informações entre os módulos sem a necessidade de sincronização. Essa abordagem é útil em sistemas onde a latência e a largura de banda são variáveis, proporcionando maior flexibilidade na comunicação.
%print of that shit mm


\subsection{Simulação e Análise}
%Simulação com token-player
%Simulação temporal
%Geração e análise do espaço de estados

\subsection{Decomposição GALS}
A decomposição GALS (Globally Asynchronous, Locally Synchronous) envolve a criação de três controladores separados que operam de forma independente, permitindo uma maior flexibilidade e escalabilidade no sistema. Essa abordagem facilita a integração de diferentes módulos e a adaptação a variações nas condições de operação.
