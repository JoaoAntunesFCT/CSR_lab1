\section{Controlador Distribuído em regime Síncrono}

\subsection{Cutting Set}
A identificação do conjunto de corte (cutting set) é um passo crucial na decomposição de sistemas distribuídos. Este conjunto consiste em um grupo de canais ou conexões que, quando removidos, dividem o sistema em subsistemas independentes. A escolha adequada do cutting set permite a implementação eficiente de controladores distribuídos, minimizando a complexidade da comunicação entre os módulos.

\subsection{Decomposição SPLIT}
A decomposição usando SPLIT (Synchronous Parallel Logic Interconnect Technology) permite a criação de uma arquitetura mais flexível e escalável, dividindo o sistema em módulos independentes que se comunicam através de canais síncronos. Essa abordagem facilita a implementação de sistemas complexos, permitindo a reutilização de componentes e a redução do tempo de desenvolvimento. A decomposição SPLIT para esta aplicação pode ser visualizada na figura abaixo.
\begin{figure}[H]
    \centering
    \includegraphics[width=0.8\textwidth]{img/06_gals.png}
    \caption{Decomposição SPLIT do sistema em módulos independentes.}
    \label{fig:split_decomposition}
\end{figure}

\subsection{Modelo com canais síncronos}
A modelação com canais síncronos envolve a definição de canais de comunicação que operam em sincronia, permitindo a troca de informações entre os módulos de forma coordenada. Essa abordagem é fundamental para garantir a consistência e a previsibilidade no comportamento do sistema. Esta modelação encontra-se representada na figura abaixo.
\begin{figure}[H]
    \centering
    \includegraphics[width=0.8\textwidth]{img/petri_sync_channels.png}
    \caption{Modelo com canais síncronos entre os módulos.}
    \label{fig:synchronous_model}
\end{figure}

\subsection{Modelo com canais assíncronos}
A modelação com canais assíncronos envolve a definição de canais de comunicação que operam de forma independente, permitindo a troca de informações entre os módulos sem a necessidade de sincronização. Essa abordagem é útil em sistemas onde a latência e a largura de banda são variáveis, proporcionando maior flexibilidade na comunicação. Este modelo está descrito na figura abaixo.
\begin{figure}[H]
    \centering
    \includegraphics[width=0.8\textwidth]{img/petri_async_channels.png}
    \caption{Modelo com canais assíncronos entre os módulos.}
    \label{fig:asynchronous_model}
\end{figure}

\subsection{Decomposição GALS}
A decomposição GALS (Globally Asynchronous, Locally Synchronous) envolve a criação de três controladores separados que operam de forma independente, permitindo uma maior flexibilidade e escalabilidade no sistema. Essa abordagem facilita a integração de diferentes módulos e a adaptação a variações nas condições de operação. A decomposição GALS para esta aplicação pode ser visualizada nas figuras abaixo.
\begin{figure}[H]
    \centering
    \begin{subfigure}[b]{0.32\textwidth}
        \centering
        \includegraphics[width=\textwidth]{img/petri_event_controler_1.png}
        \caption{Tapete 1 com controlo por eventos}
        \label{fig:gals_decomposition_1}
    \end{subfigure}\hfill
    \begin{subfigure}[b]{0.32\textwidth}
        \centering
        \includegraphics[width=\textwidth]{img/petri_event_controler_2.png}
        \caption{Tapete 2 com controlo por eventos}
        \label{fig:gals_decomposition_2}
    \end{subfigure}\hfill
    \begin{subfigure}[b]{0.32\textwidth}
        \centering
        \includegraphics[width=\textwidth]{img/petri_event_controler_3.png}
        \caption{Tapete 3 com controlo por eventos}
        \label{fig:gals_decomposition_3}
    \end{subfigure}
    \caption{Decomposição GALS — Tapetes 1–3 com controlo por eventos}
    \label{fig:gals_decomposition}
\end{figure}

\subsection{Simulação e Análise}

\subsubsection{Simulação com token-player}

abaixo estão os links para os vídeos das simulações realizadas com o token-player para os modelos com canais síncronos e assíncronos.

\vspace{1em}

\href{https://unlpt-my.sharepoint.com/:v:/g/personal/jafe_lopes_fct_unl_pt/ESR4w4p0PK9CgxJYC0eDlT8BZZrfzpxHBbuzyL5D78P0Gg?nav=eyJyZWZlcnJhbEluZm8iOnsicmVmZXJyYWxBcHAiOiJPbmVEcml2ZUZvckJ1c2luZXNzIiwicmVmZXJyYWxBcHBQbGF0Zm9ybSI6IldlYiIsInJlZmVycmFsTW9kZSI6InZpZXciLCJyZWZlcnJhbFZpZXciOiJNeUZpbGVzTGlua0NvcHkifX0&e=liF1mC}{Vídeo — decomposição do modelo com canais síncronos}

\href{https://unlpt-my.sharepoint.com/:v:/g/personal/jafe_lopes_fct_unl_pt/EWEOa2jxXRNFmaCiO5c6iG4BP21vyCbQSKYiZQ8HcoeOrg?nav=eyJyZWZlcnJhbEluZm8iOnsicmVmZXJyYWxBcHAiOiJPbmVEcml2ZUZvckJ1c2luZXNzIiwicmVmZXJyYWxBcHBQbGF0Zm9ybSI6IldlYiIsInJlZmVycmFsTW9kZSI6InZpZXciLCJyZWZlcnJhbFZpZXciOiJNeUZpbGVzTGlua0NvcHkifX0&e=j6h8N4}{Vídeo — decomposição do modelo com canais assíncronos}


\subsubsection{Geração e análise do espaço de estados}

Não pede