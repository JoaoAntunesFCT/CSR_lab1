\section{Implementação em FPGA}
Nesta secção, detalha-se o processo de implementação do sistema em uma FPGA (Field-Programmable Gate Array). 

\subsection{Geração do código VHDL}
O código VHDL encontra-se descrito na figura abaixo.
\begin{figure}[H]
    \centering
    \includegraphics[width=1\textwidth]{img/vhdl_code_fpga.png}
    \caption{Código VHDL gerado para implementação em FPGA}\label{fig:vhdl_code_fpga}
\end{figure}
Este código foi gerado automaticamente a partir do modelo de Rede de Petri utilizando a framework IOPT-Tools, que traduz as especificações do modelo para uma descrição em VHDL adequada para síntese em FPGA.

\subsection{Configuração de entradas/saídas}
Para este projeto, foram configuradas as seguintes entradas e saídas na FPGA:

\begin{figure}[H]
    \centering
    \includegraphics[width=0.8\textwidth]{img/fpga_io_configuration.png}
    \caption{Configuração de entradas e saídas na FPGA}\label{fig:fpga_io_configuration}
\end{figure}

\subsection{Testes experimentais}
%comparar resultados