\section{Buffers de Capacidade Finita}

\subsection{Modificação do modelo}
A modificação do modelo para múltiplas peças envolve a adaptação dos componentes do sistema para lidar com a entrada e saída de várias peças simultaneamente. Isso inclui a implementação de buffers de capacidade finita que permitem o armazenamento temporário das peças, garantindo que o sistema possa operar de forma eficiente mesmo quando há variações na taxa de entrada ou saída.

\begin{figure}[H]
    \centering
    \includegraphics[width=0.8\textwidth]{img/petri_tapete_singular_part8.png}
    \caption{Rede de Petri - modelo modificado para múltiplas peças (tapete único)}\label{fig:petri_inter_components}
\end{figure}

\subsection{Interconexão dos componentes}

A interconexão dos componentes no modelo modificado é realizada através de estados capazes de gerir a presença de múltiplos objetos. Garante-se assim o fluxo das peças de uma forma mais eficiente e que os buffers possam operar corretamente dentro das suas capacidades definidas.

\begin{figure}[H]
    \centering
    \includegraphics[width=0.8\textwidth]{img/petri_part8.png}
    \caption{Rede de Petri - interconexão dos diferentes tapetes}\label{fig:petri_inter_components}
\end{figure}

\subsection{Módulo de visualização do número de objetos}
%print módulo de visualização

\subsection{Simulação e Análise}

\subsubsection{Simulação com token-player}

\subsubsection{Geração e análise do espaço de estados}

\subsection{Implementação em FPGA}
%print implementação FPGA

\subsection{Análise de resultados}
%prints e análise de resultados

