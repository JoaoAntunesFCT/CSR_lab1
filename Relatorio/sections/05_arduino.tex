\section{Implementação em Arduino}

\subsection{Geração do código C}
O código C encontra-se descrito na figura abaixo.
\begin{figure}[H]
    \centering
    \includegraphics[width=1\textwidth]{img/arduino_code.png}
    \caption{Código C gerado para implementação em Arduino}\label{fig:arduino_code}
\end{figure}
Este código foi gerado automaticamente a partir do modelo de Rede de Petri utilizando a framework IOPT-Tools, que traduz as especificações do modelo para uma descrição em C adequada para execução em microcontroladores Arduino.

\subsection{Configuração de entradas/saídas}
Para este projeto, foram configuradas as seguintes entradas e saídas no Arduino:
\begin{figure}[H]
    \centering
    \includegraphics[width=0.8\textwidth]{img/arduino_io_configuration.png}
    \caption{Configuração de entradas e saídas no Arduino}\label{fig:arduino_io_configuration}
\end{figure}

\subsection{Testes}
Os testes experimentais foram realizados para validar o funcionamento do sistema implementado no Arduino. Os resultados obtidos foram comparados com os resultados da simulação para garantir a consistência e a precisão do modelo.
\begin{figure}[H]
    \centering
    \includegraphics[width=0.8\textwidth]{img/arduino_test_results.png}
    \caption{Resultados dos testes experimentais no Arduino}\label{fig:arduino_test_results}
\end{figure}

\subsection{Comparação com simulação e implementação FPGA}
%comparar resultados
