%==============================================================================
% CONFIGURAÇÃO DO DOCUMENTO
%==============================================================================
\documentclass[a4paper,12pt]{extarticle}  % Classe de documento com tamanho A4 e fonte 12pt

%==============================================================================
% PACOTES E BIBLIOTECAS
%==============================================================================

% --- Configuração de página e layout ---
\usepackage[left=30mm,top=30mm,right=30mm,bottom=30mm]{geometry}  % Margens do documento
\usepackage{etoolbox}     % Necessário para a página de capa
\usepackage{adjustbox}    % Ajustes de caixas
\usepackage{fancyhdr}     % Cabeçalho e rodapé personalizados
\setlength{\headheight}{15.05pt} % Corrige erro do fancyhdr sobre headheight
\usepackage{lastpage}     % Referência à última página

% --- Suporte a idiomas e codificação ---
\usepackage[T1]{fontenc}  % Codificação de fontes
\usepackage[utf8]{inputenc}  % Suporte a caracteres UTF-8
\usepackage[portuguese]{babel}  % Suporte ao idioma português
\usepackage{newunicodechar}
\newunicodechar{₋}{$_-$}

% --- Matemática e símbolos ---
\usepackage{amsmath}      % Funções matemáticas avançadas
\usepackage{amsfonts}     % Fontes matemáticas adicionais
\usepackage{mathtools}    % Extensões para o amsmath
\usepackage{amssymb}      % Símbolos matemáticos adicionais
\usepackage{bm}           % Símbolos matemáticos em negrito
\usepackage{siunitx}      % Formatação de unidades do SI

% --- Figuras e elementos visuais ---
\usepackage{graphicx}     % Inclusão de figuras
\usepackage{subcaption}   % Suporte a subfiguras
\usepackage{xcolor}       % Suporte a cores
\usepackage[table]{xcolor} % Cores em tabelas
\usepackage{float}        % Melhor controle do posicionamento de figuras
\usepackage{tikz}         % Criação de diagramas
\usetikzlibrary{shapes.geometric, arrows}

% --- Tabelas e listas ---
\usepackage{booktabs}     % Melhora a aparência das tabelas
\usepackage{multirow}     % Células que ocupam múltiplas linhas
\usepackage{enumitem,kantlipsum}  % Personalização de listas
\usepackage[usestackEOL]{stackengine}  % Manipulação de texto em pilha

% --- Algoritmos e código ---
\usepackage[ruled,vlined]{algorithm2e}  % Algoritmos em pseudocódigo
\usepackage{listings}     % Inclusão de código-fonte
\usepackage{minted}       % Destaque de sintaxe avançado
\usemintedstyle{emacs}    % Estilo de cores para código
\usepackage{tcolorbox}    % Caixas coloridas para código/exemplos

% --- Títulos e formatação de seções ---
\usepackage{titlesec}     % Personalização de títulos de seções
\usepackage{titletoc}     % Personalização do sumário

% --- Referências e links ---
\usepackage[square,numbers,sort]{natbib}  % Gerenciamento de bibliografia
\usepackage{hyperref}     % Hiperlinks no documento
\usepackage[capitalise]{cleveref}  % Referências cruzadas inteligentes

%==============================================================================
% CONFIGURAÇÕES PERSONALIZADAS
%==============================================================================

% --- Configuração de blocos de exemplo ---
\newtcolorbox[auto counter, number within=section]{exampleblock}[2][]{
    colframe=gray!80!black, 
    colback=white, 
    coltitle=black, 
    title=Exemplo \thetcbcounter: #2, 
    #1, 
    fonttitle=\bfseries, 
    sharp corners=southwest
}

% --- Definição de cores ---
\definecolor{azulTitulos}{RGB}{0, 76, 153}  % Define uma cor azul para títulos

% --- Formatação de títulos de seções ---
\titleformat{\section}
  {\color{azulTitulos}\normalfont\Large\bfseries}  % Seções em azul, grande e negrito
  {\thesection}{1em}{}
\titleformat{\subsection}
  {\color{azulTitulos}\normalfont\large\bfseries}  % Subseções em azul, menor e negrito
  {\thesubsection}{1em}{}

% --- Configuração de hiperlinks ---
\hypersetup{
    colorlinks,  % Ativa links coloridos
    linkcolor={black},  % Links internos em preto
    citecolor={blue!50!black},  % Citações em azul escuro
    urlcolor={blue!80!black}  % URLs em azul mais claro
}

% --- Configuração do cabeçalho e rodapé ---
\pagestyle{fancy}
\fancyhf{}
\fancyhead[L]{\leftmark}
\fancyhead[R]{\thepage}
\fancyfoot[L]{CSR}
\fancyfoot[C]{Laboratório 1}
\fancyfoot[R]{\thepage\ de \pageref{LastPage}}
\renewcommand{\headrulewidth}{0.4pt}
\renewcommand{\footrulewidth}{0.4pt}

% --- Estilos de página especiais ---
% Estilo para páginas de capítulo e primeira página
\fancypagestyle{plain}{
    \fancyhf{}
    \fancyfoot[C]{Título do Documento}
    \fancyfoot[R]{Página \thepage\ de \pageref{LastPage}}
    \renewcommand{\headrulewidth}{0pt}
    \renewcommand{\footrulewidth}{0.4pt}
}

% Estilo para primeira página (sem cabeçalho nem rodapé)
\fancypagestyle{firstpage}{
    \fancyhf{}
    \renewcommand{\headrulewidth}{0pt}
    \renewcommand{\footrulewidth}{0pt}
}

% --- Configurações gerais ---
\linespread{1}  % Espaçamento de linha simples
\newtheorem{theorem}{Theorem}[section]  % Define ambiente para teoremas
\graphicspath{{img/}}  % Pasta para figuras

% --- Listings settings ---
\lstset{
  language=Python,
  basicstyle=\ttfamily\footnotesize,
  keywordstyle=\color{blue}\bfseries,
  stringstyle=\color{red},
  commentstyle=\color{gray}\itshape,
  numbers=left,
  numberstyle=\tiny\color{gray},
  stepnumber=1,
  numbersep=8pt,
  backgroundcolor=\color{white},
  showspaces=false,
  showstringspaces=false,
  showtabs=false,
  frame=single,
  rulecolor=\color{black},
  tabsize=4,
  captionpos=b,
  breaklines=true,
  breakatwhitespace=true,
  morekeywords={self},
}


%==============================================================================
% CORPO DO DOCUMENTO
%==============================================================================
\begin{document}

%==============================================================================
% PÁGINAS PRÉ-TEXTUAIS
%==============================================================================
% --- Página de capa ---
\pagenumbering{gobble}  % Remove a numeração de páginas inicialmente
\thispagestyle{firstpage}
\newcommand{\bonecoUM}{Aniko Costa}          % Nome do primeiro docente
\newcommand{\bonecoDOIS}{Filipe Moutinho}        % Nome do segundo docente
\newcommand{\docenteTRES}{}                    % Espaço para um terceiro docente (vazio)
\newcommand{\alunoUM}{João Pedro Antunes - \textbf{70380}}      % Nome e número do terceiro aluno
\newcommand{\alunoDOIS}{Júlio Lopes- \textbf{70512}}  % Nome e número do segundo aluno
\newcommand{\judeu}{Marcos Romão - \textbf{71348}}      % Nome e número do terceiro aluno
\newcommand{\data}{1º Semestre 2025/2026}      % Período académico 
\newcommand{\cadeira}{Co-design e Sistemas Reconfiguráveis}  % cadeira
\newcommand{\titulo}{Laboratório 1}       % Título do trabalho 
\title{Título} %n mudar                        % Título alternativo (não usado na capa)
\newcommand{\departmento}{Departamento de Engenharia Eletrotécnica e de Computadores}  % Nome do departamento
\newcommand{\curso}{Mestrado em Engenharia Eletrotécnica e de Computadores}  % Nome do curso
% -------------------------------- Front content
\begin{center}\leavevmode                      % Inicia o ambiente centralizado da capa
    \normalfont
    \includegraphics[width=0.75\columnwidth]{img/logo.png}  % Logotipo da instituição
    \vskip 1cm                                 % Espaçamento vertical de 1cm
    \textsc{\Large \departmento}\\[1 cm]       % Nome do departamento em maiúsculas pequenas
    {\large \curso}                            % Nome do curso
    \vskip 1cm                                 % Espaçamento vertical de 1cm
    \rule{\linewidth}{0.2 mm}                  % Linha horizontal decorativa
    {\huge \bfseries \titulo \par}             % Título em negrito e tamanho enorme
    \vskip 1cm                                 % Espaçamento vertical de 1cm
    {\Large \bfseries \cadeira \par}          % Subtítulo em negrito e tamanho grande
           
    \rule{\linewidth}{0.2 mm}\\[1.5 cm]        % Linha horizontal decorativa com espaçamento
     
    \begin{minipage}[t]{0.45\textwidth}        % Coluna esquerda para informações dos alunos
    	\begin{flushleft} \large
    		\emph{Alunos:}\\
    		\ifdefempty{\alunoUM}{}{\alunoUM\\}     % Verifica e exibe o aluno 1 se definido
            \ifdefempty{\alunoDOIS}{}{\alunoDOIS\\} % Verifica e exibe o aluno 2 se definido
            \ifdefempty{\judeu}{}{\judeu\\}         % Verifica e exibe o aluno 3 se definido
    	\end{flushleft}
    \end{minipage}
    \begin{minipage}[t]{0.45\textwidth}        % Coluna direita para informações dos docentes
        \begin{flushright} \large
        	{\emph{Docentes:}}\\
        	\ifdefempty{\bonecoUM}{}{\bonecoUM\\}       % Verifica e exibe o docente 1 se definido
        	\ifdefempty{\bonecoDOIS}{}{\bonecoDOIS\\}   % Verifica e exibe o docente 2 se definido
        	\ifdefempty{\docenteTRES}{}{\docenteTRES\\} % Verifica e exibe o docente 3 se definido
        \end{flushright}
    \end{minipage}
    \vfill                                     % Preenche o espaço vertical restante
    {\normalsize \data\par}                    % Data do semestre acadêmico
\end{center}
\cleardoublepage                               % Finaliza a página e inicia uma nova página (ímpar)  % Inclui a página de capa

% --- Sumário e listas ---
\newpage
\pagenumbering{roman}
\setcounter{page}{1}

\tableofcontents
\newpage

% Listas opcionais (comentadas)
%\listoffigures
%\newpage
%\listoftables
%\newpage
%\listofalgorithms  % Lista de algoritmos em pseudocódigo
%\newpage
%\listoflistings  % Lista de algoritmos em formato de código
%\newpage

%==============================================================================
% CONTEÚDO PRINCIPAL
%==============================================================================
\pagenumbering{arabic}  % Inicia numeração arábica (1, 2, 3...)
\setcounter{page}{1}

% --- Seções do documento ---
\section{Introdução}

\subsection{Contextualização}
%Contextualização do problema (sistema de automação com N células em cascata)

Este trabalho insere-se no âmbito da unidade curricular de Co-design e Sistemas Reconfiguráveis, focando-se no desenvolvimento e implementação de controladores para sistemas de automação industrial. O caso de estudo considerado é um sistema de produção composto por 3 células em cascata, onde cada célula integra um braço de robô e um tapete rolante.

A complexidade destes sistemas justifica a utilização de metodologias formais de modelação e verificação, permitindo validar o comportamento do controlador antes da sua implementação física. Adicionalmente, a possibilidade de implementar o controlador de forma centralizada ou distribuída oferece diferentes trade-offs entre complexidade de implementação, desempenho e escalabilidade do sistema.

\subsection{Objetivos}
%Objetivos do trabalho
O presente trabalho tem como objetivo principal o desenvolvimento completo de um controlador para um sistema de automação com três células em cascata (N=3), desde a fase de modelação até à implementação física em hardware reconfigurável.

Os objetivos específicos incluem:

\begin{itemize}
    \item Modelar o comportamento do sistema utilizando redes de Petri Input-Output Place-Transition (IOPT), explorando as capacidades de composição modular através de operações de adição e fusão de redes;
    
    \item Validar e analisar o modelo através de simulação e análise do espaço de estados, verificando propriedades comportamentais críticas do sistema;
    
    \item Implementar o controlador em plataformas de hardware distintas (FPGA e Arduino), avaliando a abordagem centralizada de controlo;
    
    \item Desenvolver uma arquitetura de controlo distribuído através da decomposição do modelo global, explorando paradigmas de execução síncrona e GALS (Globally Asynchronous Locally Synchronous);
    
    \item Analisar o impacto de comunicações não-instantâneas entre controladores distribuídos, introduzindo atrasos pseudo-aleatórios;
    
    \item Estender o modelo para suportar buffers de capacidade finita superior a uma unidade, aumentando a flexibilidade do sistema;
    
    \item Comparar as diferentes abordagens de implementação, avaliando vantagens, desvantagens e adequação a diferentes contextos aplicacionais.
\end{itemize}
\clearpage

\section{Análise Teórica}

\subsection{Redes de Petri}
%Redes de Petri e Redes de Petri Coloridas

As redes de Petri constituem uma ferramenta formal de modelação adequada para sistemas concorrentes e distribuídos. Uma rede de Petri é definida por uma quádrupla $PN = (P, T, F, M_0)$, onde P representa lugares, T transições, F arcos direcionados, e $M_0$ a marcação inicial. A dinâmica baseia-se no disparo de transições, que removem tokens dos lugares de entrada e adicionam aos de saída.

\textbf{Redes IOPT (Input-Output Place-Transition)} estendem as redes clássicas com capacidade de interagir com o ambiente através de:
\begin{itemize}
    \item Eventos de entrada associados a sinais externos (sensores);
    \item Sinais de saída para controlo de atuadores;
    \item Guardas e prioridades para resolução de conflitos;
    \item Semântica temporal para modelar delays.
\end{itemize}

\textbf{Redes de Petri Coloridas} permitem representar de forma compacta sistemas com estrutura repetitiva. Os tokens possuem "cores" (tipos de dados), permitindo que um único modelo represente múltiplas instâncias de subsistemas idênticos. No sistema estudado, tokens $\langle 1 \rangle$, $\langle 2 \rangle$ e $\langle 3 \rangle$ identificam cada célula.

\textbf{Composição Modular:} A construção de sistemas complexos utiliza operações de:
\begin{itemize}
    \item \textit{Merge Nets}: união de múltiplas redes mantendo nós distintos;
    \item \textit{Fusion Sets}: identificação de nós a fundir criando sincronização;
    \item \textit{Net Addition}: combinação de merge e fusão num modelo integrado.
\end{itemize}

\subsection{Execução síncrona vs. GALS}
%Paradigma de execução síncrona vs. GALS (Globally Asynchronous Locally Synchronous)

Em contexto síncrono, todos os componentes operam com um relógio global comum. Oferece simplicidade de design e determinismo, mas apresenta limitações de escalabilidade, consumo energético elevado (relógio sempre ativo) e dificuldades na distribuição do sinal de relógio (clock skew) em sistemas grandes.

Já num paradigma GALS, este divide o sistema em domínios síncronos locais que comunicam assincronamente. Cada domínio opera com relógio independente, comunicando através de protocolos de handshaking ou FIFOs assíncronas.

Vantagens do GALS incluem escalabilidade, modularidade, eficiência energética e tolerância a falhas. As desvantagens são a maior complexidade de interface, latência variável na comunicação e necessidade de tratamento de metastabilidade nas fronteiras entre domínios.

\subsection{Implementação em FPGA e Arduino}
%Conceitos de implementação em FPGA e Arduino

Utilizar-se-á uma \textbf{FPGA (Field-Programmable Gate Array)}. A descrição do sistema é feita em VHDL, uma linguagem de descrição de hardware que permite especificar comportamento concorrente. Recorrer-se-á também a um microcontrolador \textbf{Arduino}.

A framework IOPT-Tools permite geração automática de código VHDL para FPGA e código C para Arduino, facilitando a comparação entre as duas abordagens de implementação.

\begin{table}[h]
\centering
\small
\begin{tabular}{|l|c|c|}
\hline
\textbf{Critério} & \textbf{FPGA} & \textbf{Arduino} \\
\hline
Paralelismo & Hardware real & Software (sequencial) \\
Tempo de resposta & ns & ms \\
\hline
\end{tabular}
\caption{Comparação entre plataformas FPGA e Arduino}
\end{table}
\clearpage

\section{Redes de Petri}

\subsection{IOPT-Tools}

\subsubsection{Rede de Petri - Tapete Singular}
\begin{figure}[H]
    \centering
    \includegraphics[width=0.8\textwidth]{img/petri_tapete_singular.png}
    \caption{Rede de Petri - Tapete Singular}
    \label{fig:petri_tapete_singular}
\end{figure}

\subsubsection{Rede de Petri - Modelo Completo}
\begin{figure}[H]
    \centering
    \includegraphics[width=0.8\textwidth]{img/petri_modelo_completo.png}
    \caption{Rede de Petri - Modelo Completo}
    \label{fig:petri_modelo_completo}
\end{figure}

\subsubsection{Rede de Petri - Fusion set}
\begin{figure}[H]
    \centering
    \includegraphics[width=0.8\textwidth]{img/petri_model_merge.png}
    \caption{Rede de Petri - Fusion set}
    \label{fig:petri_fusion_set}
\end{figure}

\subsubsection{Rede de Petri - Node set}
\begin{figure}[H]
    \centering
    \includegraphics[width=0.8\textwidth]{img/petri_node_set.png}
    \caption{Rede de Petri - Node set}
    \label{fig:petri_node_set}
\end{figure}

\subsection{Simulação e Análise}

\subsubsection{Simulação com token-player}

\subsubsection{Simulação temporal}

\subsubsection{Geração e análise do espaço de estados}

\clearpage

\section{Implementação em FPGA}

\subsection{Geração do código VHDL}

\subsection{Deployment na FPGA}

\subsection{Configuração de entradas/saídas}

\subsection{Testes experimentais}

\subsection{Comparação com resultados de simulação}
\clearpage

\section{Implementação em Arduino}

\subsection{Geração do código C}
O código C encontra-se descrito na figura abaixo.
\begin{figure}[H]
    \centering
    \includegraphics[width=1\textwidth]{img/arduino_code.png}
    \caption{Código C gerado para implementação em Arduino}\label{fig:arduino_code}
\end{figure}
Este código foi gerado automaticamente a partir do modelo de Rede de Petri utilizando a framework IOPT-Tools, que traduz as especificações do modelo para uma descrição em C adequada para execução em microcontroladores Arduino.

\subsection{Configuração de entradas/saídas}
Para este projeto, foram configuradas as seguintes entradas e saídas no Arduino:
\begin{figure}[H]
    \centering
    \includegraphics[width=0.8\textwidth]{img/arduino_io_configuration.png}
    \caption{Configuração de entradas e saídas no Arduino}\label{fig:arduino_io_configuration}
\end{figure}

\subsection{Testes}
Os testes experimentais foram realizados para validar o funcionamento do sistema implementado no Arduino. Os resultados obtidos foram comparados com os resultados da simulação para garantir a consistência e a precisão do modelo.
\begin{figure}[H]
    \centering
    \includegraphics[width=0.8\textwidth]{img/arduino_test_results.png}
    \caption{Resultados dos testes experimentais no Arduino}\label{fig:arduino_test_results}
\end{figure}

\subsection{Comparação com simulação e implementação FPGA}
%comparar resultados

\clearpage

\section{Controlador Distribuído em regime Síncrono}

\subsection{Identificação do conjunto de corte (cutting set)}

\subsection{Decomposição usando SPLIT}

\subsection{Modelo com canais síncronos}

\subsection{Simulação e validação}

\subsection{Modelo com canais assíncronos}

\subsection{Simulação e Análise}
%Simulação com token-player
%Simulação temporal
%Geração e análise do espaço de estados

\subsection{Decomposição GALS - três controladores separados}

\clearpage

\section{Implementação em FPGA - Distribuição Síncrona}

\subsection{Geração do código VHDL}
%print código VHDL

\subsection{Implementação do LFSR}
O LFSR (Linear Feedback Shift Register) é um componente essencial para a geração de números pseudoaleatórios no sistema. A sua implementação em VHDL envolve a definição dos registos de shift (deslocamento) e das funções de feedback necessárias para garantir a pseudo-aleatoriedade dos valores gerados.
Usando como referência o documento fornecido pelos docentes, procedeu-se à implementação do LFSR com as características desejadas.

Esta secção detalha o processo de implementação do LFSR, incluindo a configuração dos seus parâmetros e a integração com os restantes componentes do sistema.
\subsubsection{Gestão dos atrasos}
%Integração dos delays

\subsection{Deployment na FPGA}
%baby steps do deployment na fpga

\subsection{Testes e análise de resultados}
%print testes
%comparar resultados
\clearpage

\section{Buffers de Capacidade Finita}

\subsection{Modificação do modelo}
A modificação do modelo para múltiplas peças envolve a adaptação dos componentes do sistema para lidar com a entrada e saída de várias peças simultaneamente. Isso inclui a implementação de buffers de capacidade finita que permitem o armazenamento temporário das peças, garantindo que o sistema possa operar de forma eficiente mesmo quando há variações na taxa de entrada ou saída.

\begin{figure}[H]
    \centering
    \includegraphics[width=0.8\textwidth]{img/petri_tapete_singular_part8.png}
    \caption{Rede de Petri - modelo modificado para múltiplas peças (tapete único)}\label{fig:petri_inter_components}
\end{figure}

\subsection{Interconexão dos componentes}

A interconexão dos componentes no modelo modificado é realizada através de estados capazes de gerir a presença de múltiplos objetos. Garante-se assim o fluxo das peças de uma forma mais eficiente e que os buffers possam operar corretamente dentro das suas capacidades definidas.

\begin{figure}[H]
    \centering
    \includegraphics[width=0.8\textwidth]{img/petri_part8.png}
    \caption{Rede de Petri - interconexão dos diferentes tapetes}\label{fig:petri_inter_components}
\end{figure}

\subsection{Módulo de visualização do número de objetos}
%print módulo de visualização

\subsection{Simulação e Análise}

\subsubsection{Simulação com token-player}

\subsubsection{Geração e análise do espaço de estados}

\subsection{Implementação em FPGA}
%print implementação FPGA

\subsection{Análise de resultados}
%prints e análise de resultados


\clearpage

\section{Distribuído com Buffers (Síncrono)}

\subsection{Aplicação dos procedimentos do capítulo 06 ao modelo do capítulo 08}

\subsection{Decomposição e implementação distribuída}

\subsection{Execução síncrona global}

\subsection{Testes e análise}



\clearpage

\section{Distribuído com Buffers (Assíncrono)}

\subsection{Aplicação assíncrona ao buffer de capacidade finita}

\subsection{Comunicações não-instantâneas com buffers}

\subsection{Deployment e testes}

\subsection{Análise comparativa}


\clearpage

\section{Comparação de abordagens}

\subsection{Centralizado vs Distribuído}

\subsection{Síncrono vs Assíncrono}

\subsection{Impacto dos atrasos de comunicação} 

\subsection{Vantagens e desvantagens de cada abordagem}

\subsection{Análise de escalabilidade}  
\clearpage

\section{Conclusões}

\subsection{Resultados alcançados}
\subsection{Dificuldades encontradas e soluções adotadas}
\clearpage

%==============================================================================
% PÓS-TEXTUAIS
%==============================================================================
% --- Referências bibliográficas ---
\nocite{*}
\bibliographystyle{unsrtnat}  % Estilo de bibliografia numérico
\bibliography{bibliography}
\clearpage  % Garante que o apêndice comece em uma nova página

% --- Apêndice ---
\appendix
%\input{apendix/gnuradiopy}  % Descomente se tiver conteúdo no apêndice

\end{document}